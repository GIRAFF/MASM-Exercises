\documentclass[oneside, final, 12pt]{extarticle}
\usepackage[utf8]{inputenc}
\usepackage[russian]{babel}
\usepackage{vmargin}
\usepackage{listings}
%\usepackage{graphicx}
%\usepackage{ucs}
\setpapersize{A4}
\setmarginsrb{2cm}{2cm}{2cm}{2cm}{0pt}{0mm}{0pt}{13mm}
\usepackage{indentfirst}		%красная строка
%\usepackage{color}
\sloppy

\begin{document}
\begin{titlepage}
	\begin{centering}
		\textsc{Министерство образования и науки Российской Федерации}\\
		\textsc{Новосибирский государственный технический университет}\\
		\textsc{Кафедра теоритической и прикладной информатики}\\
	\end{centering}
	%\centerline{\hfill\hrulefill\hrulefill\hfill}
	\vfill
	\vfill
	\vfill
	\Large
	\centerline{Лабораторная работа №1}
	\centerline{по дисциплине "<Низкоуровневое программирование">}
	\normalsize
	\vfill
	\vfill
	\vfill
	\begin{flushleft}
		\begin{minipage}{0.3\textwidth}
			\begin{tabular}{l l}
				Факультет: & ПМИ\\
				Группа: & ПМИ-41\\
				Студент: & Кислицын И. О.\\
				Преподаватель: & Лисицин Д. В.
			\end{tabular}
		\end{minipage}
	\end{flushleft}
	\vfill
	\vfill
	\begin{centering}
		Новосибирск\\
		2016\\
	\end{centering}
\end{titlepage}
\setcounter{page}{2}
\lstset{
	breaklines=\true,
	%frame=single,
	basicstyle=\footnotesize\ttfamily,
	tabsize=2,
	showspaces=\false,
	breaklines=\true,
	breakatwhitespace=\true,
	escapeinside={"}{"},
	%inputencoding=utf8x,
	extendedchars=\true,
	keepspaces=\true
}
\section{Цель работы}
Изучить и приобрести практические навыки работы с основными командами языка Ассемблера, функциями ввода-вывода, регистрами и символьными данными.
\section{Задача}
Разработать программу которая принимает на вход два числа в восьмеричной системе счисления, вычисляет их разность и выводит результат в шестнадцатеричной форме.
\section{Текст программы}
\lstinputlisting{../lab1.asm}
\section{Тесты}
\begin{tabular}{|c|c|c|} \hline
	\bf№ & \bf in & \bf out \\ \hline
	1 & \(300_8\) \(300_8\) & \(0\) \\ \hline
	2 & \(300_8\) \(-300_8\) & \(180_{16}\) \\ \hline
	3 & \(0_8\) \(100_8\) & \(-40_{16}\) \\ \hline
	4 & \(777_8\) \(0_8\) & \(1FF_{16}\) \\ \hline
\end{tabular}
\end{document}
